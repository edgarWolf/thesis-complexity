\chapter{Einleitung}
Die Komplexitätstheorie beschäftigt sich mit der \enquote{Schwierigkeit} von Problemen, im Sinne vom Platzverbrauch sowie Zeitverbrauch. Sie unterteilt dabei die Probleme in Komplexitätsklassen ein, 
um so ihre Schwierigkeit miteineander vergleichen zu können. Ein entscheidendes Konzept ist dabei der Nichtdeterminismus, sodass Entscheidungsprobleme, für die kein deterministischer Polynomialzeit-Algorithmus bekannt ist,
über diese Klasse abgebildet werden. Es wird vermutet dass P echt in der Klasse NP enthalten ist, und die Probleme aus NP sich tatsächlich nicht deterministisch effizient entscheiden lassen.
Durch die Klasse coNP werden Probleme beschrieben, deren Komplement in NP liegen. Diese Probleme erscheinen nochmals schwieriger als die Probleme in NP, und auch hier stellt sich die Frage, ob coNP = NP.
Es werden aber nicht alle natürlichen Probleme durch diese drei Klassen charakterisiert. Probleme, die sicht nicht ausschließlich nichtdeterministisch entscheiden lassen, werden mittels diesen Klassen nicht hinreichend charakterisiert.
Ein Konzept, welches diese Probleme entsprechend erfasst ist die sog. Komplexitätsklasse \emph{Polynomialzeithierarchie}, die die Probleme zwischen NP und PSPACE beschreibt. 

\section{Forschungsfragen}
Die Polynomialzeithierarchie ist ein Forschungsgegenstand der Komplexitätstheorie, der bisweilen über einige offene Fragen verfügt. 
So ist nicht bekannt, ob diese Hierarchie über endlich viele Stufen verfügt, oder es unendlich viele Abstufungen innerhalb dieser Hierarchie gibt.
Auch ist nicht klar, ob die Hierarchie über echte Teilmengen definiert ist, was im gegenteiligen Fall auch eine endliche Anzahl von Abstufungen impliziert.
Im Kern besteht die Frage, ob Verallgemeinerungen des Nichtdeterminismus tatsächlich eine erhöhte Mächtigkeit im Kontext der Berechenbarkeit mit sich bringen, oder dies nur scheinbar der Fall ist.

\section{Ziel der Arbeit}
Die Arbeit hat das Ziel einen Überblick über den aktuellen Forschungsstand über die Polynomialzeithierarchie zu erbringen. Dabei werden Kenntnisse über Konzepte wie der Nichtdeterminismus, sowie grundlegende Begriffe der Komplexitätstheorie
wie die Komplexitätsklassen P, NP, und coNP, aber auch die platzbasierte Komplexitätsklasse PSPACE vorausgesetzt, sodass die Ausführungen dieser Arbeit auf diesen Wissensstand aufbauen können.
Es werden die Vermutungen über die Eigenschaften der Polynomialzeithierarchie näher beleuchtet und erläutert, wieso die in der Forschung dominierenden Vermutungen plausibler erscheinen, als die Gegenannahme.

\section{Struktur der Arbeit}
Zunächst wird das Konzept der Alternierung eingeführt, anhand einer Analogie veranschaulicht, und auf Turingmaschinen angewendet. In diesem Zuge werden alternierende Komplexitätsklassen definiert, und der Zusammenhang 
zwischen der Klasse AP und PSPACE dargestellt. Anschließend wird auf Grundlage dieses Wissens die Polynomialzeithierarchie formal eingeführt, sowie über die verschiedenen Definitionsmöglichkeiten definiert.
Es werden dabei die Gemeinsamkeiten und Unterschiede der Definitionen genannt, sowie erläutert, aus welchem Grund diese Definitionen äquivalent sind.
Zum Abschluss wird auf die Eigenschaften der Polynomialzeithierarchie eingegangen, sowie die Vermutungen der Forschung aufgezeigt. 
Es werden die Konsequenzen der gegenteiligen Aussagen bezüglich den Vermutungen skizziert und erklärt, warum dies nicht überzeugend erscheint.
Ein Fazit fasst alle Erkenntnisse aus den bisherigen Abschnitten zusammen und schließt die Arbeit mit einem Ausblick bezüglich weiterer möglicher Forschungsergebnisse ab.

\input{./Bestandteile/Kapitel/einordnung}
\chapter{Alternierung} \label{chapter: Alternierung}

\section{Begriffsklärung} \label{section: Begriffsklärung}
Die Alternierung in der Komplexitätstheorie meint eine Verallgemeinerung des Nichtdeterminusmus. Mittels dieser Verallgemeinerung 
können diverse Beweise simplifiziert werden, und Verbindungen zu anderen Komplexitätsklassen gezogen werden \cite{sipser_introduction_2012} \cite{chandra_alternation_1976}. Ein Algorithmus, der auf Alternierung basiert,
kann wie ein gewöhnlicher nichtdeterministischer Algorithmus bei jedem Berechnungsschritt in mehrere Zustände überführt werden. Der Unterschied liegt hier bei dem Modus der Akzeptanz:
Während ein nichtdeterministischer Algorithmus akzeptiert, wenn mindestens ein Berechnugspfad akzeptiert, so tritt bei einem alternierenden Algorithmus eine Fallunterscheidung ein.
Die aktuelle Berechnug innerhalb des Algorithmus akzeptiert dann entweder, wenn wie beim Nichtdeterminusmus, mindestens ein Berechnugspfad akzeptiert, oder wenn alle Berechnugspfade akzeptieren.
Entscheidend für den Modus der Akzeptanz der Berechnung ist die Kennzeichnung des Berechnungsschrittes, der die Information enthält, unter welchem Modus dieser akzeptiert \cite{sipser_introduction_2012}.
Diese Kennzeichnung wird in der Literatur meist über die logischen Quantoren $\exists$ und $\forall$ vorgenommen, manche Autoren wie Sipser verwenden die Symbole $\land$ und $\lor$, für die Darstellung als logisches Und bzw. als logisches Oder.
Im ersten Fall akzeptiert die Berechnung, wenn mindestens ein Kind-Berechnugspfad akzeptiert, analog akzeptiert ein universeller Zustand genau dannn, wenn alle Kind-Berechnugspfade akzeptieren.
Eine Analogie hierfür könnte ein Prozess in einem Computer sein, der bei jedem Berechnungsschritt mehrere neue Prozesse erzeugt.
Jeder dieser Prozesse enthält eine Information darüber, ob er eine existenzielle oder universelle Berechnug durchführt. Der erzeugende Prozess gibt nun selber $1$ zurück, wenn er sich in einem existenziellen Zustand befindet
und mindestens ein Kind-Prozess mit $1$ antwortet, oder er sich in einem universellen Prozess befindet und alle Prozesse mit $1$ antworten \cite{sipser_introduction_2012}.
Der Nichtdeterminismus kann als Spezialfall der Alternierung betrachtet werden, bei dem Berechungen ausschließlich existenziell durchgeführt werden \cite{chandra_alternation_1976}.

\section{Alternierende Turingmaschinen} \label{section: alternierende Turingmaschine}
Das Konzept der Alternierung lässt sich auf Turingmaschinen übetragen, wodurch sich folgende Definition ergibt:
\begin{definition}[Alternierende Turingmaschine (ATM)]\cite{pass_lecture_nodate}
    Eine alternierende Turingmaschine ATM ist eine nichtdeterministische Turingmaschine NDTM $M$, bei der jeder Zustand mit einem 
    Element aus der Menge \{$\forall$, $\exists$, \texttt{accept}, \texttt{halt}\} gekennzeichnet ist.
\end{definition}

Eine ATM kann dabei genau wie eine NDTM bei jedem Berechnungsschritt in mehrere Zustände übergehen, jedoch operiert diese nicht
wie die NDTM zwangsläufig in einem existenziellen Modus, sodass sie akzeptiert, wenn es einen akzeptierenden Berechungspfad gibt, sondern entsprechend der Markierung des Zustandes.
Der Name \enquote{alternierende Turingmaschine} kommt daher, dass die Turingmaschine zwischen dem Existenz- und Universalqunator alternieren kann.
Die Akzeptanz einer Eingabe bei einer ATM kann formal wie folgt definiert werden:
\begin{definition}[Akzeptanz einer alternierenden Turingmaschine]\cite{arora_computational_2009}
    Sei $G_{M, x}$ der Konfigurationsgraph der ATM $M$ auf die Eingabe $x \in \{0,1\}*$, sodass eine Kante von einer Konfiguration $C$ zu 
    einer Konfiguration $C'$ einen Übergang von $C$ zu $C'$ über die Transitionsfunktion von $M$ darstellt. Dann ist die Akzeptanz über folgenden
    Markierungsalgorithmus definiert:

    \begin{itemize}
        \item Markiere die Konfigurationen terminierend in einem akzeptierenden Zustand $C_{accept}$ mit \texttt{ACCEPT}.
    \end{itemize}
    \begin{itemize}        
        \item Wenn eine Konfiguration $C$ mit $\exists$ markiert ist, und es eine Kante von $C$ zu $C'$ mit Markierung \texttt{ACCEPT} gibt, markiere $C$ mit \texttt{ACCEPT}.
        \item Wenn eine Konfiguration $C$ mit $\forall$ markiert ist, und alle Kanten von $C$ zu Konfigurationen $C'$ mit Markierung \texttt{ACCEPT} führen, markiere $C$ mit \texttt{ACCEPT}.
        \item Wiederhole die obigen Schritte, bis keine weitere Markierung mehr möglich ist.
    \end{itemize}

    Die ATM M akzeptiert genau dann, wenn am Ende dieses Algorithmus die Startkonfiguration $C_{start}$ mit \texttt{ACCEPT} markiert ist.
\end{definition}
\section{Alternierende Komplexitätsklassen} \label{section: Komplexitätsklassen bei alternierenden TMs}
Aus dieser hinzugekommenen Art der Turingmaschinenen ergeben sich entsprechende Komplexitätsklassen,
um die Komplexität dieser Turingmaschinen formal beschreiben zu können:
\begin{definition}[Alternierende Komplexitätsklassen]
    % Check: Muss hier nicht O Notation?
    $ATIME(f(n))$ ist die Menge Menge aller Sprachen, die von einer $f(n)$-zeitbeschränkten alternierenden Turingmaschine entschieden werden können. \\
    $ASPACE(f(n))$ ist die Menge aller Sprachen, die von einer $f(n)$-platzbeschränkten alternierenden Turingmaschine entschieden werden können.
\end{definition}
Anhand dieser allgemeinenen Definition der Komplexitätsklassen ergeben sich die konkreten Pendants zu den bisher bekannten Klassen:
% TODO: Hier vllt über die Vereinugng der ATIME Klassen gehen, ist formal korrekter
\begin{definition}[Die Klasse $AP$]
    $$AP = \bigcup_{c \in \mathbb{N}} ATIME(n^c)$$
    $AP$ ist die Menge aller Sprachen, die von einer polynomiell-zeitbeschränkten Turingmaschine entschieden werden können.
\end{definition}
Besonders für diese Klasse ergibt sich ein interessanter Zusammenhang zur platzbasierten Komplexitätsklasse $PSPACE$:

\begin{theorem}
    $$AP = PSPACE$$
\end{theorem}

\begin{proof}[Beweis]
    Für den Beweis muss die Inklusion in beide Richtungen gezeigt werden, also $AP \subseteq PSPACE$ und $PSPACE \subseteq AP$. \\
    Die erste Inklusion folgt aus dem $PSPACE$-vollständigen Problem $TQBF$, bei dem eine mit Quantoren versehene logische Formel auf Erfüllbarkeit geprüft wird.
    Eine alternierende Turingmaschine $M$ kann einfach jede Belegung einer mit $\exists$ versehenen Variable existenziell raten, analog jede Belegung einer mit $\forall$ 
    versehenen Variable universell.\\
    Für die zweite Inklusion konstruieren wir eine Turingmaschine $M'$, die die gleiche Sprache wie $M$ entscheidet.
    % TODO: Beweis zu Ende führen und Zitate.
\end{proof}
\chapter{Polynomialzeithierarchie (PH)} \label{chapter: Polynomialzeithierarchie (PH)}
Die Polynomialzeithierarchie (kurz PH) bildet eine Verallgemeinerung der Klassen P, NP, und coNP an, also den Klassen, die einer 
polynomiellen Zeitbeschränkung unterliegen \cite{aaronson_scott_2016}. Die PH bietet einen Formalismus, um auch diese Probleme adäquat zu untersuchen und sie auf ihre Komplexität hin bewerten zu können.
Die PH verfügt dabei über verschiedene Definitionen, die aber allesamt äquivalent sind.
% TODO: Hier wenn Zeit und Platz das mit EXACT-INDSET oder so einfügen


\section{Definition mit alternierenden Quantoren} \label{section: Definition PH mit alternierenden Quantoren}
Eine Möglichkeit der Definition besteht übr die der alternierenden Quantoren. Es wird dabei zwischen zwei Sprachdefinitionen unterschieden.
Sei dafür $p$ ein Polynom und $i \geq 1$. Dann ist eine Sprache $L$ in $\Sigma^p_i$, wenn es polynomiell zeitbeschränkte Turingmaschine $M$ gibt, 
sodass für jede Eingabe $x \in \{0,1\}^*$ gilt \cite{rossman_complexity_2015}:
%todo: Die Erklärung der Quantifizierer und wann welcher Qunator hergenommen wir hinzufügen.%
\begin{align*}
    x \in L \Leftrightarrow \exists u_1 \in \{0,1\}^{p(|x|)} \forall u_2 \in \{0,1\}^{p(|x|)} ... Q_i u_i \in \{0,1\}^{p(|x|)} : M(x, u_1, ..., u_i) = 1
\end{align*}
wobei $Q_i = \forall$, wenn $i$ gerade ist, sonst $Q_i = \exists$. \\
Analog ist eine Sprache $L$ in $\Pi^p_i$, wenn es eine polynomiell zeitbeschränkte Turingmaschine $M$ gibt, sodass für jede Eingabe $x \in \{0,1\}^*$ gilt \cite{arora_computational_2009}:
\begin{align*}
    x \in L \Leftrightarrow \forall u_1 \in \{0,1\}^{p(|x|)} \exists u_2 \in \{0,1\}^{p(|x|)} ... Q_i u_i \in \{0,1\}^{p(|x|)} : M(x, u_1, ..., u_i) = 1
\end{align*}
wobei $Q_i = \exists$, wenn $i$ gerade ist, sonst $Q_i = \forall$.
Aus diesen Definitionen kann unmittelbar gefolgert werden dass $\Sigma^p_1 = NP$ und $\Pi^p_1 = coNP$, denn die boolsche Formel reduziert auf 
den ersten Quantoren entspricht genau der Definition der entsprechenden Komplexitätsklassen $NP$ bzw. $coNP$.
Allgemein ist sogar $\Pi^p_{i+1} = \text{co}\Sigma^p_{i+1}$, was einem Wechsel der Quantoren entspricht.
Die Polynomialzeithierarchie entspricht dann der Vereinigung über alle $\Sigma^p_i$, sowie der Vereinigung aller $\Pi^p_i$ \cite{sipser_introduction_2012}:
\begin{align*}
    \text{PH} = \bigcup_{i \geq 1} \Sigma^p_i = \bigcup_{i \geq 1} \Pi^p_i 
\end{align*}
\section{Definition mit alternierenden Turingmaschinen} \label{section: Definition PH mit ATM}
Eine äquivalente und mit der Definition über die alternierenden Quantoren verwandte Definition ist über alternierende Turingmaschinen möglich.
Das Konzept wurde bereits in \ref{section: alternierende Turingmaschine} erläutert und dient nun als Basis für die Definition der Polynomialzeithierarchie.
Sei dafür $M$ eine $f(n)$ zeitbeschränkte ATM und $i \geq 1$  mit $i - 1$ mit  Alternierungen. Dann entscheidet M die Sprache $\Sigma_i TIME(f(n))$, wenn der initiale Zustand ein existenzieller ist,
oder die Sprache $\Pi_i TIME(f(n))$, falls der erste Zustand ein universeller ist \cite{arora_computational_2009}.
Die Klassen $\Sigma^p_i$ und $\Pi^p_i$ sind dann definiert als \cite{arora_computational_2009}:
\begin{align*}
    \Sigma^p_i = \bigcup_{c \in \mathbb{N}} \Sigma_i \text{TIME}(n^c) \\
    \Pi^p_i = \bigcup_{c \in \mathbb{N}} \Pi_i \text{TIME}(n^c)
\end{align*}
$\Sigma^p_i$ beschreibt also die Menge der Sprachen, die von einer polynomiell zeitbeschränkten ATM $M$ mit existenziellem Startzustand und höchstens $i-1$ Alternierungen entschieden werden.
$\Pi^p_i$ beschreibt analog die Menge der Sprachen, die von einer polynomiell zeitbeschränkten ATM $M$ mit universellem Startzustand und höchstens $i-1$ Alternierungen entschieden werden.
Die PH selbst ist wie in \ref{section: Definition PH mit alternierenden Quantoren} definiert als die Vereinigung aller $\Sigma^p_i$ und $\Pi^p_i$.

\section{Definition mit Orakel-Turingmaschinen} \label{section: Definition PH mit Orakel-Turingmaschinen}
Eine weitere Definition erfolgt über Turingmaschinen mit Zugriff auf ein Orakel. Dabei wird auf ein Orakel zugegriffen,
das ein vollständiges Problem einer Komplexitätsklasse entscheiden kann. Sei für die folgenden Ausführungen $\text{A}^\text{B}$, sodass eine A eine Komplexitätsklasse ist, deren Sprachen durch eine Turingmaschine mit Zugriff auf ein Orakel, welches alle Probleme aus B entscheiden kann, entscheidet. 
Dann ist die Polynomialzeithierarchie induktiv, für $i \geq 0$ definiert als \cite{rothe_komplexitatstheorie_2008}:
\begin{align*}
    & \Delta^p_0 = \Sigma^p_0 = \Pi^p_0 = \text{P} \\
    & \Delta^p_{i+1} = \text{P}^{\Sigma^p_i} \\
    &\Sigma^p_{i+1} = \text{NP}^{\Sigma^p_i} \\
    & \Pi^p_{i+1} = \text{coNP}^{\Sigma^p_i}, \Pi^p_{i+1} = \text{co}\Sigma^p_{i+1} \\
    & \text{PH} = \bigcup_{k \geq 0} \Sigma^p_k
\end{align*}
Diese Definition beschreibt induktiv die erhöhte Berechnungskraft je Stufe der PH durch die Nutzung eines weiteren Orakels.
Es folgt aus den obigen Definitionen, dass $\Delta^p_1 = \text{P}^\text{P} = \text{P}$.
Ein weiterer Unterschied zu den anderen zwei Definitionen besteht über die Komplexitätsklassen $\Delta^p_i$, die die Komplexitätsklasse P
mit Zugriff auf Orakel betrachtet. Außerdem wird noch eine weitere Stufe $0$ der Hierarchie betrachtet, bei der alle Klassen gleich P sind.
Jedoch spielt die Klassen $\Delta^p_i$ in der strukturellen Betrachtung der Polynomialzeithierarchie keine große
Rolle und sind für die weiteren Ausführungen von nebensächlicher Relevanz. \\

\noindent Die obigen Definitionen sind äquivalent, was gezeigt werden kann, indem gezeigt wird dass $\Sigma^p_{i+1} = \text{NP}^{\Sigma^p_i \text{SAT}}$.
Dafür wird zuerst gezeigt, dass $\Sigma^p_2 \subseteq \text{NP}^{\text{SAT}}$. Sei dafür $L \in \Sigma^p_2$. Dann gilt: 
\begin{align*}
    x \in L \Leftrightarrow \exists u_1 \forall u_2 : M(x, u_1, u_2) = 1
\end{align*}
Dann kann $u_1$ nichtdeterministisch geraten werden, um dann das Orakel, welches SAT entscheiden kann, zu fragen, ob es ein $u_2$ gibt, sodass das Orakel die Eingabe nicht akzeptiert.
$M$ akzeptiert genau dann, wenn das Orakel nicht akzeptiert. \\
Nun wird gezeigt, dass $\text{NP}^{\text{SAT}} \subseteq \Sigma^p_2 $. Dann kann $L$ entschieden werden, indem es nichtdeterministisch Entscheidungen $y$ gibt, 
sowie Orakel-Anfragen $q_1, ..., q_k$ und Orakel-Antworten $a_1, ..., a_k$ sodass $M$ $x$ in polynomieller Zeit akzeptiert. Zusätzlich muss sichergestellt sein, 
dass die Antworten $a_i$ korrekte Antworten bezüglich der Anfragen $q_i$ sind, also dass $a_i = 1$ genau dann wenn  $q_i \in \text{SAT}$.
\chapter{Bezug der PH zur Komplexitätstheorie} \label{chapter: Bezug der PH zur Komplexitätstheorie}
% TODO: Vllt umbennenn in "Eigenschaften der PH"
\section{Vollständige Mengen innerhalb der PH} \label{section: Vollständige Mengen innerhalb der PH}

\section{Kollabieren der PH} \label{section: Kollabieren der PH}
\chapter{Fazit} \label{chapter: Fazit}
Es scheint so, dass die erhöhte Mächtigkeit in der Berechnung definiert über die Polynomialzeithierarchie tatsächlich existiert.
Die Tatsache Berechnungen über mehrere Stufen alternierend durchführen zu können, und damit die Begrenzungen des Nichtdeterminismus zu überwinden,
oder Zugriff auf ein Orakel zu haben, welches vollständige Probleme einer Komplexitätsklasse lösen kann, wirken mächtiger als Klassen, die diese Fähigkeit nicht haben.
Nichtsdestotrotz sind das lediglich Vermutungen und dieses Themengebiet ist noch nicht in aller Vollständigkeit erforscht, als dass diverse Fragestellungen noch nicht beantwortet sind.
Sollte sich herausstellen, dass die Polynomialzeithierarchie kollabiert, so bedeutet dies, dass die in der Hierarchie definierte zusätzliche Berechnungsmacht 
in Wahrheit keine ist, sodass im Extremfall die komplette Hierarchie in sich auf die Komplexitätsklasse P zusammenbricht. In diesem Zuge wäre somit auch die Frage ob  P $=$ NP gelöst,
mit allen Konsequenzen. Dieser Fall ist allerdings nach den aktuellen Forschungskenntnissen unplausibel, es wird also vermutet, dass die Hierarchie nicht kollabiert, und die in der Arbeit formulierten Konzepte
die Berechnungsmacht echt erweitern.  
