\chapter{Einleitung}
Eine der zentralen Fragen der Komplexitätstheorie ist, ob P $=$ NP gilt.  Es wird vermutet dass P echt in der Klasse NP enthalten ist, und für die Probleme aus NP tatsächlich keine deterministische Algorithmen in Polynomialzeit existieren, die diese Probleme entscheiden \cite{sipser_introduction_2012}.
Das Gegenteil würde bedeuten, dass alle Probleme aus NP ebenfalls deterministisch in Polynomialzeit entschieden werden können, mit weitreichenden Auswirkungen für Berechnungen in unserer Welt \cite{arora_computational_2009}.
Durch die Klasse coNP werden Probleme beschrieben, deren Komplemente in NP liegen. Diese Probleme erscheinen nochmals schwieriger als die Probleme in NP, und auch hier stellt sich die Frage, ob coNP $=$ NP.
Es werden aber nicht alle natürlichen Probleme durch diese drei Klassen charakterisiert. Probleme, die sich nicht ausschließlich nichtdeterministisch entscheiden lassen, werden mittels dieser Klassen nicht hinreichend charakterisiert.
Probleme wie z.B. das Problem \texttt{EXACT-INDSET}, bei dem mit einer Eingabe eines Graphen $G$ und einer Zahl $k \in \mathbb{N}$ entschieden werden soll, ob das größte \texttt{INDSET} in $G$ genau die Größe $k$ hat,
scheinen über kein \enquote{kurzes} Zertifikat zu verfügen \cite{arora_computational_2009}. Es scheint außerdem so, dass dieses Problem nicht allein nichtdeterministisch gelöst werden kann, als dass die Existenz eines \texttt{INDSET} entschieden werden muss, zugleich aber sichergestellt sein muss, dass es kein \texttt{INDSET} der Größe $l > k$ gibt \cite{arora_computational_2009}. 
Ein Konzept, welches diese Probleme entsprechend erfasst, ist die sog. \emph{Polynomialzeithierarchie}, die als Komplexitätsklasse die Probleme \enquote{zwischen} P bzw. NP und PSPACE beschreibt \cite{arora_computational_2009}. 

\section{Forschungsfragen}
Die Polynomialzeithierarchie ist ein Forschungsgegenstand der Komplexitätstheorie, der bisweilen über einige offene Fragen verfügt. 
So ist nicht bekannt, ob diese Hierarchie über endlich viele Stufen verfügt, oder ob es unendlich viele Abstufungen innerhalb dieser Hierarchie gibt.
Auch ist nicht klar, ob die Hierarchie über echte Teilmengen definiert ist, was im gegenteiligen Fall auch eine endliche Anzahl von Abstufungen impliziert.
Im Kern besteht die Frage, ob der Nichtdeterminismus und seine Verallgemeinerungen tatsächlich eine erhöhte Mächtigkeit im Kontext der Berechenbarkeit mit sich bringen, oder ob dies nur scheinbar der Fall ist.

\section{Ziel der Arbeit}
Die Arbeit hat das Ziel einen Überblick über den aktuellen Forschungsstand über die Polynomialzeithierarchie zu erbringen. Dabei werden Kenntnisse über Konzepte, wie den Nichtdeterminismus, den grundlegende Begriffen der Komplexitätstheorie
wie die Komplexitätsklassen P, NP, und coNP,  aber auch der platzbasierten Komplexitätsklasse PSPACE vorausgesetzt. Ebenso werden Kenntnisse über Turingmaschinen mit Zugriff auf ein Orakel erwartet, sodass die Ausführungen dieser Arbeit auf diesem Wissensstand aufbauen können, ohne all diese Begrifflichkeiten erneut zu definieren.
Es werden die Vermutungen über die Eigenschaften der Polynomialzeithierarchie näher beleuchtet und erläutert, wieso diese plausibler erscheinen, als die gegenteilige Annahme.

\section{Struktur der Arbeit}
Zunächst wird das Konzept der \emph{Alternierung} eingeführt, anhand einer Analogie veranschaulicht, und auf Turingmaschinen angewendet. In diesem Zuge werden alternierende Komplexitätsklassen definiert, und der Zusammenhang 
zwischen der Klasse AP und PSPACE dargestellt. Anschließend wird auf Grundlage dieses Wissens die Polynomialzeithierarchie formal eingeführt, sowie über die verschiedenen Definitionsmöglichkeiten definiert.
Zum Abschluss wird auf die Möglichkeit eines Kollabierens der Polynomialzeithierarchie eingegangen.
Dabei werden verschiedene Szenarien eines solchen Kollaps skizziert und deren Folgen erörtert.
Ein Fazit fasst alle Erkenntnisse aus den bisherigen Abschnitten zusammen und geht abschließend auf die Konsequenzen des möglichen Ausgangs ungeklärter Fragestellungen ein.

\chapter{Alternierung} \label{chapter: Alternierung}

\section{Begriffsklärung} \label{section: Begriffsklärung}
Die Alternierung in der Komplexitätstheorie meint eine Verallgemeinerung des Nichtdeterminusmus. Mittels dieser Verallgemeinerung 
können diverse Beweise simplifiziert werden, und Verbindungen zu anderen Komplexitätsklassen gezogen werden \cite{sipser_introduction_2012} \cite{chandra_alternation_1976}. Ein Algorithmus, der auf Alternierung basiert,
kann wie ein gewöhnlicher nichtdeterministischer Algorithmus bei jedem Berechnungsschritt in mehrere Zustände überführt werden. Der Unterschied liegt hier bei dem Modus der Akzeptanz:
Während ein nichtdeterministischer Algorithmus akzeptiert, wenn mindestens ein Berechnugspfad akzeptiert, so tritt bei einem alternierenden Algorithmus eine Fallunterscheidung ein.
Die aktuelle Berechnug innerhalb des Algorithmus akzeptiert dann entweder, wenn wie beim Nichtdeterminusmus, mindestens ein Berechnugspfad akzeptiert, oder wenn alle Berechnugspfade akzeptieren.
Entscheidend für den Modus der Akzeptanz der Berechnung ist die Kennzeichnung des Berechnungsschrittes, der die Information enthält, unter welchem Modus dieser akzeptiert \cite{sipser_introduction_2012}.
Diese Kennzeichnung wir in der Literatur meist über die logischen Quantoren $\exists$ und $\forall$ vorgenommen, manche Autoren wie Sipser verwenden die Symbole $\land$ und $\lor$, für die Darstellung als logisches Und bzw. als logisches Oder.
Im ersten Fall akzeptiert die Berechnung, wenn mindestens ein Kind-Berechnugspfad akzeptiert, analog akzeptiert ein universeller Zustand genau dannn, wenn alle Kind-Berechnugspfade akzeptieren.
Eine Analogie hierfür könnte ein Prozess in einem Computer sein, der bei jedem Berechnungsschritt mehrere neue Prozesse erzeugt.
Jeder dieser Prozesse enthält eine Information darüber, ob er eine existenzielle oder universelle Berechnug durchführt. Der erzeugende Prozess gibt nun selber $1$ zurück, wenn er sich in einem existenziellen Zustand befindet
und mindestens ein Kind-Prozess mit $1$ antwortet, oder er sich in einem universellen Prozess befindet und alle Prozesse mit $1$ antworten \cite{sipser_introduction_2012}.
Wir können daher den Nichtdeterminusmus als einen Spezalfall der Alternierung auffassen, bei dem Berechungen ausschließlich existenziell durchgeführt werden \cite{chandra_alternation_1976}.

\section{Alternierende Turingmaschinen} \label{section: alternierende Turingmaschine}
Das Konzept der Alternierung lässt sich auf Turingmaschinen übetragen, wodurch sich folgende Definition ergibt:
\begin{definition}[Alternierende Turingmaschine (ATM)]\cite{pass_lecture_nodate}
    Eine alternierende Turingmaschine ATM ist eine nichtdeterministische Turingmaschine NDTM $M$, bei der jeder Zustand mit einem 
    Element aus der Menge \{$\forall$, $\exists$, \texttt{accept}, \texttt{halt}\} gekennzeichnet ist.
\end{definition}

Eine ATM kann dabei genau wie eine NDTM bei jedem Berechnungsschritt in mehrere Zustände übergehen, jedoch operiert diese nicht
wie die NDTM zwangsläufig in einem existenziellen Modus, sodass sie akzeptiert, wenn es einen akzeptierenden Berechungspfad gibt, sondern entsprechend der Markierung des Zustandes.
Der Name \enquote{alternierende Turingmaschine} kommt daher, dass die Turingmaschine zwischen dem Existenz- und Universalqunator alternieren kann.
Die Akzeptanz einer Eingabe bei einer ATM kann formal wie folgt definiert werden:
\begin{definition}[Akzeptanz einer alternierenden Turingmaschine]\cite{arora_computational_2009}
    Sei $G_{M, x}$ der Konfigurationsgraph der ATM $M$ auf die Eingabe $x \in \{0,1\}*$, sodass eine Kante von einer Konfiguration $C$ zu 
    einer Konfiguration $C$ einen Übergang von $C$ zu $C'$ über die Transitionsfunktion von $M$ darstellt. Dann ist die Akzeptanz über folgenden
    Markierungsalgorithmus definiert:

    \begin{itemize}
        \item Markiere die Konfigurationen terminierend in einem akzeptierenden Zustand $C_{accept}$ mit \texttt{ACCEPT}.
    \end{itemize}
    Wiederhole
    \begin{itemize}        
        \item Wenn eine Konfiguration $C$ mit $\exists$ markiert ist, und es eine Kante von $C$ zu $C'$ mit Markierung \texttt{ACCEPT} gibt, markiere $C$ mit \texttt{ACCEPT}.
        \item Wenn eine Konfiguration $C$ mit $\forall$ markiert ist, und alle Kanten von $C$ zu Konfigurationen $C'$ mit Markierung \texttt{ACCEPT} führen, markiere $C$ mit \texttt{ACCEPT}.
    \end{itemize}
    bis keine Markierung mehr möglich ist.

    Die ATM M akzeptiert genau dann, wenn am Ende dieses Algorithmus die Startkonfiguration $C_{start}$ mit \texttt{ACCEPT} markiert ist.
\end{definition}
\section{Alternierende Komplexitätsklassen} \label{section: Komplexitätsklassen bei alternierenden TMs}
\chapter{Polynomialzeithierarchie (PH)} \label{chapter: Polynomialzeithierarchie (PH)}
Die Komplexitätsklassen $P$, $NP$, und $coNP$ decken einige natürliche Entscheidungsprobleme nicht ab, die aber auch 
nicht $EXPTIME$ zuzuordnen sind. Für diese Probleme wurde die Polynomialzeithierarchie (abkürzend PH) eingeführt, sodass 
auch diese Probleme entsprechend auf ihre Komplexität untersucht werden können.
Die PH verfügt dabei über verschiedene Definitionen, die aber allesamt äquivalent sind.
% TODO: Hier wenn Zeit und Platz dasa mit EXACT-INDSET oder so einfügen


\section{Definition mit alternierenden Turingmaschinen} \label{section: Definition PH mit ATM}

\section{Definition mit alternierenden Quantoren} \label{section: Definition PH mit alternierenden Quantoren}

\section{Definition mit Orakel-Turingmaschinen} \label{section: Definition PH mit Orakel-Turingmaschinen}
% TODO: Umbennnen in "Vollständigkeit" mit erstrem kapitel = Definition
\chapter{Eigenschaften der PH} \label{chapter: Eigenschaften der PH}
Nachdem die Polynomialzeithierarchie im vorigen Abschnitt über drei verschiedene Definitionen eingeführt wurde, beschäftigt sich dieser Abschnitt mit 
weiteren Eigenschaften der PH und den in der Forschung formulierten Vermutungen, sowie den Konsequenzen, falls sich diese Vermutungen als falsch herausstellen sollten.

\section{Vollständigkeit} \label{section: Vollständige Mengen innerhalb der PH}
Die Vollständigkeit innerhalb der Polynomialzeithierarchie ist analog zu der Vollständigkeit der klassischen polynomiell beschränkten Klassen
P, NP, coNP und PSPACE definiert \cite{arora_computational_2009}:

\begin{definition}[$\Sigma^p_i$-Vollständigkeit] \cite{arora_computational_2009}
    Eine Sprache $L$ ist $\Sigma^p_i$-vollständig, genau dann wenn $L \in \Sigma^p_i$ und wenn gilt: 
    $$
    \forall L' \in \Sigma^p_i : L' \leq_p L
    $$
\end{definition}

Die Definition für $\Pi^p_i$-Vollständigkeit ergibt sich genau wie die für $\Sigma^p_i$ analog.
Für den Fall $i = 1$ entsprechen diese Definitionen genau der Definition der NP- bzw. coNP-Vollständigkeit.
Eine ähnliche Überlegung kann für vollständige Probleme der gesamten PH gemacht werden:

\begin{definition}[PH-Vollständigkeit] \cite{arora_computational_2009}
    Eine Sprache $L$ ist PH-vollständig, genau dann wenn $L \in \text{PH}$ und wenn gilt 
    $$
    \forall L' \in \text{PH} : L' \leq_p L
    $$
\end{definition}

Es wird allerdings vermutet, dass keine solchen vollständigen Mengen für die gesamte Hierarchie existent sind, und sich die Vollständigkeit auf Probleme innerhalb der
Stufen der PH beschränken. Die Konsequenzen für das Gegenteil dieser Annahme werden in \ref{section: Kollabieren der PH} betrachtet, diese werden als unplausibel gewertet \cite{arora_computational_2009}.
Generell gilt, dass jede Stufe der PH als vollständiges Problem eine spezielle Ausprägung des \texttt{TQBF}-Problems als vollständiges Problem besitzt.
Das \texttt{TQBF}-Problem sei dabei folgendermaßen definiert:
\begin{definition}[\texttt{TQBF}] \cite{sipser_introduction_2012}
    \textbf{Gegeben:} Eine quantifizierte boolsche Formel $\phi$ 
    $$
    Q_1 u_1 Q_2 u_2, ... \phi(u_1, u_2, ...)
    $$ \\
    wobei $Q_i \in \{\exists, \forall\}$
    \textbf{Frage:} Ist $Q_1 u_1 Q_2 u_2, ... \phi(u_1, u_2, ...) = 1$?
\end{definition}
Das Problem ist als PSPACE-vollständig bekannt und stellt eine Verallgemeinerung des \texttt{SAT}-Problems dar \cite{arora_computational_2009}. \\
Nun wird $\Sigma_i$-\texttt{SAT} definiert:
\begin{definition}[$\Sigma_i$-\texttt{SAT}] \cite{arora_computational_2009}
    \textbf{Gegeben:} Eine quantifizierte boolsche Formel $\phi$ mit $i-1$ Alternierungen der Quantoren $Q_i$ 
    $$
    \exists u_1 \forall u_2, ..., Q_i u_i \phi(u_1, u_2, ..., u_i)
    $$
    wobei $Q_i = \exists $ falls $i$ ungerade ist, $Q_i = \forall$ sonst \\
    \textbf{Frage:} Ist $\exists u_1 \forall u_2, ..., Q_i u_i \phi(u_1, u_2, ..., u_i) = 1$ ?
\end{definition}

Analog erfolgt die Definition für $\Pi^p_i$-\texttt{SAT}. Die Abgrenzung zu \texttt{TQBF} erfolgt über die Limitierung der Alternierungen für die Quantoren auf eine feste Anzahl $i$, während 
die Anzahl der Alternierungen in \texttt{TQBF} variabel sind.
Diese Entscheidungsprobleme sind $\Sigma^p_i$- und $\Pi^p_i$-vollständig, jedes Problem aus $\Sigma^p_i$- bzw. $\Pi^p_i$ kann darauf reduziert werden.
Schafer und Umans haben einen Artikel verfasst, der vollständige Probleme ab der zweiten Stufe auflistet und die Komplexität dieser kommentiert \cite{schaefer_completeness_nodate}.


\section{Kollabieren} \label{section: Kollabieren der PH}
Es wird vermutet, dass sich die Hierarchie über unendlich viele Stufen verfügt und eine Stufe $i$ in Stufe $i + 1$ echt enthalten ist,
also dass $\Sigma^p_i \subset \Sigma^p_{i+1}$ und $\Pi^p_i \subset \Pi^p_{i+1}$ für $i \geq 1$ \cite{arora_computational_2009}. Diese Vermutung ist aber nicht bewiesen und stellt eine offene
Forschungsfrage dar. Eine weitere Vermutung, bereits in \ref{section: Vollständige Mengen innerhalb der PH} artikuliert,  ist, dass es zwar vollständige Probleme je Stufe in der Polynomialzeithierarchie gibt, aber keine vollständige Mengen für die gesamte PH.
\begin{theorem}
    Sei $L$ eine PH-vollständige Sprache. Dann kollabiert die PH auf eine endliche Anzahl von Stufen. 
\end{theorem}

\begin{proof}[Beweis] \cite{arora_computational_2009}
    Da $L$ PH-vollständig ist, sind alle Sprachen aus PH auf diese Sprache $L$ reduzierbar:
    $$
    \forall L' \in PH: L' \leq_p L
    $$
    Sei nun L in der Stufe $i$, $i \geq 0$, also $L \in \Sigma^p_i$. Da nun alle Sprachen auf $L$ reduziert werden können, insbesondere
    auch Sprachen aus den Stufen $j > i$, sind auch diese Sprachen in Stufe i enthalten.
    Das heißt, es gilt: 
    $$
    PH \subseteq \Sigma^p_i
    $$ 
    und das auch für alle Klassen $\Sigma^p_j, j> i$.
    Aus dieser Inklusion folgt somit direkt, dass die PH nur $i$ Stufen besitzt, und auf die Stufe $i$ kollabiert.
\end{proof}

Eine weitere Möglichkeit, sodass die PH kollabiert, besteht dann, wenn eine für eine Stufe $\Sigma^p_i = \Pi^p_i$ gilt:

\begin{theorem}
    Sei $p$ ein Polynom und $\Sigma^p_i = \Pi^p_i$ für eine Stufe $i, i \geq 0$. Dann kollabiert die PH auf die Stufe $i$.
\end{theorem}

\begin{proof}[Beweis] \cite{rothe_komplexitatstheorie_2008}
    Nach Voraussetzung gilt für alle Sprachen $L$ in $\Sigma^p_{i+1}$, sodass für jede Eingabe $x \in \{0,1\}^*$ eine Funktion mit 
    $$
    x \in L \Leftrightarrow \exists u_1 \in \{0,1\}^{p(|x|)} \forall u_2 \in \{0,1\}^{p(|x|)} ... Q_{i+1}u_{i + 1} \in \{0,1\}^{p(|x|)} : M(x, u_1, ..., u_{i+1}) = 1
    $$
    existiert.
    Dabei ist zu beachten, dass der Teil der Formel nach $\exists u_1$ eine Sprache $L' \in \Pi^p_i$ ist.
    Das heißt für die Sprache $L$:
    \begin{align*}
    x \in L \Leftrightarrow \exists u_1 \in {0,1}^{p(|x|)} \langle x, u_1 \rangle \in L'
    \end{align*}
    Nach der Annahme, dass  $\Sigma^p_i = \Pi^p_i$ gibt ist also auch $L' \in \Sigma^p_i$, womit gilt:
    \begin{align*}
    \langle x, u_1 \rangle \in L' \Leftrightarrow \exists v_1 \forall v_2 ... Q_i v_i : M'(\langle x, u_1 \rangle, v_1, ..., v_i) = 1 
    \end{align*}
    Eingesetzt in die Definition der Sprache $L$ folgt daraus:
    \begin{align*}
    & x \in L \Leftrightarrow \exists u_1 \langle x, u_1 \rangle \in L' \\
    & \Leftrightarrow \exists u_1 (\exists v_1, \forall v_2, ..., Q_i v_i :  M'(\langle x, u_1 \rangle, v_1, ..., v_i) = 1) \\
    & \Leftrightarrow \exists \langle u_1, v_1 \rangle \forall v_2, ..., Q_i v_i : M'(\langle x, u_1 \rangle, v_1, ..., v_i) = 1
    \end{align*}
    Somit ist $L \in \Sigma^p_i$, aber zu Beginn des Beweises wurde $L$ beliebig aus $\Sigma^p_{i+1}$ gewählt, sodass nun $\Sigma^p_i = \Sigma^p_{i+1}$ gezeigt ist.
\end{proof}


Sollte die PH kollabieren, so sind die Konzepte um die Berechnungskraft zu erhöhen in Wahrheit vergebens, denn die Probleme lassen sich in diesem bereits effizienter entscheiden.
Im Extremfall würde das bedeuten P $=$ NP, sodass die gesamte Hierarchie sich auf die Klasse P beschränkt, und alle Probleme effizient deterministisch lösbar sind.
Je kleiner das $i$ gewählt wird, sodass die Hierarchie kollabiert, umso weniger glaubwürdig erscheint diese Annahme.
Gerade der Fall P $=$ NP scheint so unplausibel, dass die Annahme fortbesteht, dass es sich bei den Inklusionen in der PH um echte Teilmengen handelt.

\chapter{Fazit} \label{chapter: Fazit}
Es scheint so, dass die erhöhte Mächtigkeit in der Berechnung, definiert über die Polynomialzeithierarchie tatsächlich existiert.
Die Tatsache Berechnungen über mehrere Stufen alternierend durchführen zu können, und damit die Begrenzungen des Nichtdeterminismus zu überwinden,
oder Zugriff auf ein Orakel zu haben, welches vollständige Probleme einer Komplexitätsklasse lösen kann, wirk mächtiger als Klassen, die diese Fähigkeit nicht haben.
Nichtsdestotrotz sind dies lediglich Vermutungen und dieses Themengebiet ist noch nicht in aller Vollständigkeit erforscht.
Sollte sich herausstellen, dass die Polynomialzeithierarchie kollabiert, so bedeutet dies, dass die in der Hierarchie definierte zusätzliche Berechnungsmacht 
in Wahrheit keine ist, sodass sich im Extremfall die komplette Hierarchie im Fall von P $=$ NP auf die Komplexitätsklasse P zusammenbricht. Die Konsequenzen der Antwort auf die Frage nach dem P-NP-Problem hat folglich solch weitreichende Folgen, dass diese über das Verhältnis dieser zwei Klassen hinausgehen.
