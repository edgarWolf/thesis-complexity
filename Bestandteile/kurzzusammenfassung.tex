\newpage

\vspace*{1cm}

\begin{center}
    \textbf{Kurzzusammenfassung}
\end{center}

\vspace*{1cm}

\noindent 
Die Polynomialzeithierarchie, kurz PH, beschreibt die Hierarchie der polynomiell zeitbeschränkten Komplexitätsklassen, je nach Definition beginnend bei P oder NP.
Die gesamte Hierarchie ist dabei in PSPACE enthalten, wobei die Frage, ob PH = PSPACE nicht geklärt ist.
Die Definition dieser Hierarchie erfolgt dabei über mehrere Arten, etwa über den Mechanismus der Alternierung, einer Verallgemeinerung des Nichtdeterminusmus, oder aber über Turingmaschinen mit Zugriff auf ein Orkal, welches 
Probleme aus polynomiell zeitbeschränkten Klassen entscheiden kann. Die Definitionen sind dabei äquivalent und beschreiben gleichermaßen diese Hierarchie. Es scheint, dass die Abstufungen
dieser Hierarchie echte Inklusionen sind, ein Beweis hierfür ist aber nicht erbracht. Jedoch wäre die gegenteilige Annahme in dem Sinne unplausibiel, als dass die zusätzliche Mächtigkeit in der Berechnug, die 
die Alternierung bzw. die Orakel miteinbringen, eigentlich keine echte erhöhte Mächtigkeit bewirken. Die Folge wäre, dass die Hierarchie kollabiert, und auf eine endliche Anzahl an Stufen zusammenbricht. Sollte sie auf die unterste Stufe fallen, so wäre P = NP,
und die Hierarchie würde als Gesamtes kollabieren. In jedem Fall wäre aber die Konsequenz bei einem Kollabieren, dass PH = PSPACE, und alle PSPACE-vollständigen Mengen wären somit auch PH-vollständig.