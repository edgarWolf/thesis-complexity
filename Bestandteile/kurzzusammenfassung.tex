\newpage

\vspace*{1cm}

\begin{center}
    \textbf{Kurzzusammenfassung}
\end{center}

\vspace*{1cm}

\noindent 
Diese Arbeit im Rahmen des Seminars \enquote{Komplexität} soll einen Überblick über das Thema der Polynomialzeithierarchie, kurz PH geben. Dabei werden die wichtigsten inhaltlichen Aspekte zusammengefasst
und die Konzepte sowie Beweise näher beleuchtet. Nach einer Einführung in das Thema durch eine Einordnung des Begriffs im Kontext der Komplexitätsklassen und einer anschließenden Aufbereitung der 
verschiedenen Definitionen, wird das Konzept der alternierenden Turingmaschinen aufbereitet. Im Zuge dessen wird der Zusammenhang der Gleichheit der Klassen AP und PSPACE gezeigt. Schlussendlich 
wird auf ein mögliches Kollabieren der Hierarchie eingegangen und die Bedeutung dieses Falls. Abschließend wird eine Zusammenfassung über den Stand der Forschung formuliert, sowie ein Ausblick auf die Relevanz
weiterer Erkenntnisse in diesem Themengebiet.