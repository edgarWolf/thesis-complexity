\newpage

\vspace*{1cm}

\begin{center}
    \textbf{Kurzzusammenfassung}
\end{center}

\vspace*{1cm}

\noindent 
Die Polynomialzeithierarchie, kurz PH, beschreibt die Hierarchie der polynomiell zeitbeschränkten Komplexitätsklassen, je nach Definition beginnend bei P oder NP.
Die gesamte Hierarchie ist dabei in PSPACE enthalten, wobei die Frage, ob PH = PSPACE nicht geklärt ist.
Die Definition dieser Hierarchie erfolgt dabei über mehrere Arten, etwa über den Mechanismus der Alternierung, einer Verallgemeinerung des Nichtdeterminismus, oder aber über Turingmaschinen mit Zugriff auf ein Orakel, welches 
Probleme aus polynomiell zeit-beschränkten Klassen entscheiden kann. Die Definitionen sind dabei äquivalent und beschreiben gleichermaßen diese Hierarchie. Es scheint, dass die Abstufungen
dieser Hierarchie echte Inklusionen sind, ein Beweis hierfür ist aber nicht erbracht. Jedoch wäre die gegenteilige Annahme in dem Sinne unplausibel, als dass die Hierarchie kollabieren würde, wodurch die zusätzliche Mächtigkeit in der Berechnung, die 
die Alternierung bzw. die Orakel in der Folge haben, eigentlich keine echte erhöhte Mächtigkeit bewirken. Sollte P $=$ NP bewiesen werden, so wäre auch in diesem Fall ein Beweis dafür erbracht, dass die gesamte Polynomialzeithierarchie kollabiert, und auf die Komplexitätsklasse P in sich zusammenbricht.