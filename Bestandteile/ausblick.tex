\chapter{Fazit} \label{chapter: Fazit}
Es scheint so, dass die erhöhte Mächtigkeit in der Berechnung definiert über die Polynomialzeithierarchie tatsächlich existiert.
Die Tatsache Berechnungen über mehrere Stufen alternierend durchführen zu können, und damit die Begrenzungen des Nichtdeterminismus zu überwinden,
oder Zugriff auf ein Orakel zu haben, welches vollständige Probleme einer Komplexitätsklasse lösen kann, wirken mächtiger als Klassen, die diese Fähigkeit nicht haben.
Nichtsdestotrotz sind das lediglich Vermutungen und dieses Themengebiet ist noch nicht in aller Vollständigkeit erforscht, als dass diverse Fragestellungen noch nicht beantwortet sind.
Sollte sich herausstellen, dass die Polynomialzeithierarchie kollabiert, so bedeutet dies, dass die in der Hierarchie definierte zusätzliche Berechnungsmacht 
in Wahrheit keine ist, sodass im Extremfall die komplette Hierarchie in sich auf die Komplexitätsklasse P zusammenbricht. In diesem Zuge wäre somit auch die Frage ob  P $=$ NP gelöst,
mit allen Konsequenzen. Dieser Fall ist allerdings nach den aktuellen Forschungskenntnissen unplausibel, es wird also vermutet, dass die Hierarchie nicht kollabiert, und die in der Arbeit formulierten Konzepte
die Berechnungsmacht echt erweitern.  