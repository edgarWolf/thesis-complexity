\chapter{Fazit} \label{chapter: Fazit}
Es scheint so, dass die erhöhte Mächtigkeit in der Berechnung, definiert über die Polynomialzeithierarchie tatsächlich existiert.
Die Tatsache Berechnungen über mehrere Stufen alternierend durchführen zu können, und damit die Begrenzungen des Nichtdeterminismus zu überwinden,
oder Zugriff auf ein Orakel zu haben, welches vollständige Probleme einer Komplexitätsklasse lösen kann, wirk mächtiger als Klassen, die diese Fähigkeit nicht haben.
Nichtsdestotrotz sind dies lediglich Vermutungen und dieses Themengebiet ist noch nicht in aller Vollständigkeit erforscht.
Sollte sich herausstellen, dass die Polynomialzeithierarchie kollabiert, so bedeutet dies, dass die in der Hierarchie definierte zusätzliche Berechnungsmacht 
in Wahrheit keine ist, sodass sich im Extremfall die komplette Hierarchie im Fall von P $=$ NP auf die Komplexitätsklasse P zusammenbricht. Die Konsequenzen der Antwort auf die Frage nach dem P-NP-Problem hat folglich solch weitreichende Folgen, dass diese über das Verhältnis dieser zwei Klassen hinausgehen.