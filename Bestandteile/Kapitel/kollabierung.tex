\chapter{Eigenschaften der PH} \label{chapter: Eigenschaften der PH}
Nachdem die Polynomialzeithierarchie im vorigen Abschnitt über drei verschiedene Definitionen eingeführt wurde, beschäftigt sich dieser Abschnitt mit 
weiteren Eigenschaften der PH und den in der Forschung formulierten Vermutungen, sowie den Konsequenzen, falls sich diese Vermutungen als falsch herausstellen sollten.

\section{Vollständige Mengen innerhalb der PH} \label{section: Vollständige Mengen innerhalb der PH}

\section{Kollabieren der PH} \label{section: Kollabieren der PH}
Es wird vermutet, dass sich die Hierarchie über unendlich viele Stufen verfügt und eine Stufe $i$ in Stufe $i + 1$ echt enthalten ist,
also dass $\Sigma^p_i \subset \Sigma^p_{i+1}$ und $\Pi^p_i \subset \Pi^p_{i+1}$ für $i \geq 1$. Diese Inklusion ist aber nicht bewiesen und stellt eine offene
Forschungsfrage dar. Eine weitere Vermutung ist, dass es zwar vollständige Probleme je Stufe in der Polynomialzeithierarchie gibt, aber keine vollständige Probleme der gesamten PH.
\begin{theorem}
    Sei $L$ eine PH-vollständige Sprache. Dann kollabiert die PH auf eine endliche Anzahl von Stufen. 
\end{theorem}

\begin{proof}[Beweis]
    Da $L$ PH-vollständig ist, sind alle Sprachen aus PH auf diese Sprache $L$ reduzierbar:
    $$
    \forall L' \in PH: L' \leq_p L
    $$
    Sei nun L in der Stufe $i$, $i \geq 0$, also $L \in \Sigma^p_i$. Da nun alle Sprachen auf $L$ reduziert werden können, insbesondere
    auch Sprachen aus den Stufen $j > i$, sind auch diese Sprachen in Stufe i enthalten.
    Das heißt, es gilt: 
    $$
    PH \subseteq \Sigma^p_i
    $$ 
    und das auch für alle Klassen $\Sigma^p_j, j> i$.
    Aus dieser Inklusion folgt somit direkt, dass die PH nur $i$ Stufen besitzt, und auf die Stufe $i$ kollabiert.
\end{proof}

Eine weitere Möglichkeit, sodass die PH kollabiert, besteht dann, wenn eine für eine Stufe $\Sigma^p_i = \Pi^p_i$ gilt:

\begin{theorem}
    Sei $p$ ein Polynom und $\Sigma^p_i = \Pi^p_i$ für eine Stufe $i, i \geq 0$. Dann kollabiert die PH auf die Stufe $i$.
\end{theorem}

\begin{proof}[Beweis]
    Der Behauptung folgt direkt aus der Tatsache, dass $\Sigma^p_i = \Pi^p_i \Rightarrow \Sigma^p_{i+1} = \Pi^p_{i+1} = \Sigma^p_i$, es genügt also das zu zeigen.
    Nach Voraussetzung gilt für alle Sprachen $L$ in $\Sigma^p_{i+1}$, sodass für jede Eingabe $x \in \{0,1\}^*$ eine Funktion mit 
    $$
    x \in L \Leftrightarrow \exists u_1 \in \{0,1\}^{p(|x|)} \forall u_2 \in \{0,1\}^{p(|x|)} ... Q_{i+1}u_{i + 1} \in {0,1}^{p(|x|)} \in \{0,1\}^* : M(x, u_1, ..., u_{i+1}) = 1
    $$
    existiert.
    Dabei ist zu beachten, dass der Teil der Formel nach $\exists u_1$ eine Sprache $L' \in \Pi^p_i$ ist.
    Das heißt für die Sprache $L$:
    \begin{align*}
    x \in L \Leftrightarrow \exists u_1 \in {0,1}^{p(|x|)} \langle x, u_1 \rangle \in L'
    \end{align*}
    Nach der Annahme, dass  $\Sigma^p_i = \Pi^p_i$ gibt ist also auch $L' \in \Sigma^p_i$, womit gilt:
    \begin{align*}
    \langle x, u_1 \rangle \in L' \Leftrightarrow \exists v_1 \forall v_2 ... Q_i v_i : M'(\langle x, u_1 \rangle, v_1, ..., v_i) = 1 
    \end{align*}
    Eingesetzt in die Definition der Sprache $L$ folgt daraus:
    \begin{align*}
    & x \in L \Leftrightarrow \exists u_1 \langle x, u_1 \rangle \in L' \\
    & \Leftrightarrow \exists u_1 (\exists v_1, \forall v_2, ..., Q_i v_i :  M'(\langle x, u_1 \rangle, v_1, ..., v_i) = 1) \\
    & \Leftrightarrow \exists \langle u_1, v_1 \rangle \forall v_2, ..., Q_i v_i : M'(\langle x, u_1 \rangle, v_1, ..., v_i) = 1
    \end{align*}
    Somit ist $L \in \Sigma^p_i$. Da $\Sigma^p_i = \Pi^p_i$ und $\Sigma^p_i = co\Pi^p_i$ und $\Pi^p_i = co\Sigma^p_i$ muss also gelten:
    \begin{align*}
    \Pi^p_{i+1} = co\Pi^p_{i+1} = co\Sigma^p_i = \Pi^p_i = \Sigma^p_i
    \end{align*}
\end{proof}