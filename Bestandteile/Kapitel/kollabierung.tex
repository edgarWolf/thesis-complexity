\chapter{Eigenschaften der PH} \label{chapter: Eigenschaften der PH}
Nachdem die Polynomialzeithierarchie im vorigen Abschnitt über drei verschiedene Definitionen eingeführt wurde, beschäftigt sich dieser Abschnitt mit 
weiteren Eigenschaften der PH und den in der Forschung formulierten Vermutungen, sowie den Konsequenzen, falls sich diese Vermutungen als falsch herausstellen sollten.

\section{Vollständigkeit} \label{section: Vollständige Mengen innerhalb der PH}
Die Vollständigkeit innerhalb der Polynomialzeithierarchie ist analog zu der Vollständigkeit der klassischen polynomiell beschränkten Klassen
P, NP, coNP und PSPACE definiert \cite{arora_computational_2009}:

\begin{definition}[$\Sigma^p_i$-Vollständigkeit] \cite{arora_computational_2009}
    Eine Sprache $L$ ist $\Sigma^p_i$-vollständig, genau dann wenn $L \in \Sigma^p_i$ und wenn gilt: 
    $$
    \forall L' \in \Sigma^p_i : L' \leq_p L
    $$
\end{definition}

Die Definition für $\Pi^p_i$-Vollständigkeit ergibt sich genau wie die für die $\Sigma^p_i$ analog.
Für den Fall $i = 1$ entsprechen diese Definitionen genau der Definition der NP- bzw. coNP-Vollständigkeit.
Eine ähnliche Überlegung kann für vollständige Probleme der gesamten PH gemacht werden, sodass sich folgende Definition daraus ergibt:

\begin{definition}[PH-Vollständigkeit] \cite{arora_computational_2009}
    Eine Sprache $L$ ist PH-vollständig, genau dann wenn $L \in \text{PH}$ und wenn gilt 
    $$
    \forall L' \in \text{PH} : L' \leq_p L
    $$
\end{definition}

Es wird allerdings vermutet, dass keine solchen vollständigen Mengen für die gesamte Hierarchie existent sind, und sich die Vollständigkeit auf Probleme innerhalb der
Stufen der PH beschränken. Die Konsequenzen für das Gegenteil dieser Annahme werden in \ref{section: Kollabieren der PH} betrachtet, diese werden als unplausibel gewertet \cite{arora_computational_2009}.
Generell gilt, dass jede Stufe der PH als vollständiges Problem eine spezielle Ausprägung des \texttt{TQBF}-Problems als vollständiges Problem besitzt.
Dafür werden folgende Probleme definiert:

\begin{definition}[$\Sigma_i$-\texttt{SAT}] \cite{arora_computational_2009}
    \textbf{Gegeben:} Eine quantifizierte boolsche Formel $\phi$ mit $i-1$ Alternierungen der Quantoren $Q_i$ 
    $$
    \exists u_1 \forall u_2, ..., Q_i u_i \phi(u_1, u_2, ..., u_i)
    $$
    wobei $Q_i = \exists $ falls $i$ ungerade ist, $Q_i = \forall$ sonst. \\
    \textbf{Frage:} Ist $\exists u_1 \forall u_2, ..., Q_i u_i \phi(u_1, u_2, ..., u_i) = 1$ ?
\end{definition}

Analog erfolgt die Definition für $\Pi^p_i$-\texttt{SAT}. Die Abgrenzung zu \texttt{TQBF} erfolgt über die Limitierung der Alternierungen für die Quantoren.
Diese Entscheidungsprobleme sind $\Sigma^p_i$- und $\Pi^p_i$-vollständig, jedes Problem aus $\Sigma^p_i$- bzw. $\Pi^p_i$ kann darauf reduziert werden.
Schafer und Umans haben im Jahr 2008 einen Artikel verfasst, der vollständige Probleme ab der 2. Stufe auflistet und die Komplexität dieser kommentiert \cite{schaefer_completeness_nodate}.


\section{Kollabieren} \label{section: Kollabieren der PH}
Es wird vermutet, dass sich die Hierarchie über unendlich viele Stufen verfügt und eine Stufe $i$ in Stufe $i + 1$ echt enthalten ist,
also dass $\Sigma^p_i \subset \Sigma^p_{i+1}$ und $\Pi^p_i \subset \Pi^p_{i+1}$ für $i \geq 1$. Diese Inklusion ist aber nicht bewiesen und stellt eine offene
Forschungsfrage dar. Eine weitere Vermutung, bereits in \ref{section: Vollständige Mengen innerhalb der PH} artikuliert,  ist, dass es zwar vollständige Probleme je Stufe in der Polynomialzeithierarchie gibt, aber keine vollständige Mengen für die gesamte PH.
\begin{theorem}
    Sei $L$ eine PH-vollständige Sprache. Dann kollabiert die PH auf eine endliche Anzahl von Stufen. 
\end{theorem}

\begin{proof}[Beweis]
    Da $L$ PH-vollständig ist, sind alle Sprachen aus PH auf diese Sprache $L$ reduzierbar:
    $$
    \forall L' \in PH: L' \leq_p L
    $$
    Sei nun L in der Stufe $i$, $i \geq 0$, also $L \in \Sigma^p_i$. Da nun alle Sprachen auf $L$ reduziert werden können, insbesondere
    auch Sprachen aus den Stufen $j > i$, sind auch diese Sprachen in Stufe i enthalten.
    Das heißt, es gilt: 
    $$
    PH \subseteq \Sigma^p_i
    $$ 
    und das auch für alle Klassen $\Sigma^p_j, j> i$.
    Aus dieser Inklusion folgt somit direkt, dass die PH nur $i$ Stufen besitzt, und auf die Stufe $i$ kollabiert.
\end{proof}

Eine weitere Möglichkeit, sodass die PH kollabiert, besteht dann, wenn eine für eine Stufe $\Sigma^p_i = \Pi^p_i$ gilt:

\begin{theorem}
    Sei $p$ ein Polynom und $\Sigma^p_i = \Pi^p_i$ für eine Stufe $i, i \geq 0$. Dann kollabiert die PH auf die Stufe $i$.
\end{theorem}

\begin{proof}[Beweis]
    Der Behauptung folgt direkt aus der Tatsache, dass $\Sigma^p_i = \Pi^p_i \Rightarrow \Sigma^p_{i+1} = \Pi^p_{i+1} = \Sigma^p_i$, es genügt also das zu zeigen.
    Nach Voraussetzung gilt für alle Sprachen $L$ in $\Sigma^p_{i+1}$, sodass für jede Eingabe $x \in \{0,1\}^*$ eine Funktion mit 
    $$
    x \in L \Leftrightarrow \exists u_1 \in \{0,1\}^{p(|x|)} \forall u_2 \in \{0,1\}^{p(|x|)} ... Q_{i+1}u_{i + 1} \in \{0,1\}^{p(|x|)} : M(x, u_1, ..., u_{i+1}) = 1
    $$
    existiert.
    Dabei ist zu beachten, dass der Teil der Formel nach $\exists u_1$ eine Sprache $L' \in \Pi^p_i$ ist.
    Das heißt für die Sprache $L$:
    \begin{align*}
    x \in L \Leftrightarrow \exists u_1 \in {0,1}^{p(|x|)} \langle x, u_1 \rangle \in L'
    \end{align*}
    Nach der Annahme, dass  $\Sigma^p_i = \Pi^p_i$ gibt ist also auch $L' \in \Sigma^p_i$, womit gilt:
    \begin{align*}
    \langle x, u_1 \rangle \in L' \Leftrightarrow \exists v_1 \forall v_2 ... Q_i v_i : M'(\langle x, u_1 \rangle, v_1, ..., v_i) = 1 
    \end{align*}
    Eingesetzt in die Definition der Sprache $L$ folgt daraus:
    \begin{align*}
    & x \in L \Leftrightarrow \exists u_1 \langle x, u_1 \rangle \in L' \\
    & \Leftrightarrow \exists u_1 (\exists v_1, \forall v_2, ..., Q_i v_i :  M'(\langle x, u_1 \rangle, v_1, ..., v_i) = 1) \\
    & \Leftrightarrow \exists \langle u_1, v_1 \rangle \forall v_2, ..., Q_i v_i : M'(\langle x, u_1 \rangle, v_1, ..., v_i) = 1
    \end{align*}
    Somit ist $L \in \Sigma^p_i$. Da $\Sigma^p_i = \Pi^p_i$ und $\Sigma^p_i = co\Pi^p_i$ und $\Pi^p_i = co\Sigma^p_i$ muss also gelten:
    \begin{align*}
    \Pi^p_{i+1} = co\Pi^p_{i+1} = co\Sigma^p_i = \Pi^p_i = \Sigma^p_i
    \end{align*}
\end{proof}

%TODO: Hier P = NP einbauen.%
Sollte die PH kollabieren, so sind die Konzepte um die Berechnungskraft zu erhöhen in Wahrheit vergebens, denn die Probleme lassen sich in diesem bereits effizienter entscheiden.
Im Extremfall würde das bedeuten P $=$ NP, sodass die gesamte Hierarchie sich auf die Klasse P beschränkt, und alle Probleme effizient deterministisch lösbar sind.
Gerade der Fall P $=$ NP scheint so unplausibel, dass das Kollabieren der PH als unwahrscheinlich angesehen wird, und die Annahme fortbesteht, dass es sich bei den Inklusionen in der PH um echte Teilmengen handelt.
