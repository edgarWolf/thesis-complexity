\chapter{Eigenschaften der PH} \label{chapter: Eigenschaften der PH}
% TODO: Vllt umbennenn in "Eigenschaften der PH"
\section{Vollständige Mengen innerhalb der PH} \label{section: Vollständige Mengen innerhalb der PH}

\section{Kollabieren der PH} \label{section: Kollabieren der PH}
Es wird vermutet, dass sich die Hierarchie über unendlich viele Stufen verfügt und eine Stufe $i$ in Stufe $i + 1$ echt enthalten ist,
also dass $\Sigma^p_i \subset \Sigma^p_{i+1}$ und $\Pi^p_i \subset \Pi^p_{i+1}$ für $i \geq 1$. Diese Inklusion ist aber nicht bewiesen und stellt eine offene
Forschungsfrage dar. Eine weitere Vermutung ist, dass es zwar vollständige Probleme je Stufe in der Polynomialzeithierarchie gibt, aber keine vollständige Probleme der gesamten PH.
\begin{theorem}
    Sei $L$ eine PH-vollständige Sprache. Dann kollabiert die PH auf eine endliche Anzahl von Stufen. 
\end{theorem}

\begin{proof}[Beweis]
    Da $L$ PH-vollständig ist, sind alle Sprachen aus PH auf diese Sprache $L$ reduzierbar:
    $$
    \forall L' \in PH: L' \leq_p L
    $$
    Sei nun L in der Stufe $i$, $i \geq 0$, also $L \in \Sigma^p_i$. Da nun alle Sprachen auf $L$ reduziert werden können, insbesondere
    auch Sprachen aus den Stufen $j > i$, sind auch diese Sprachen in Stufe i enthalten.
    Das heißt, es gilt: 
    $$
    PH \subseteq \Sigma^p_i
    $$ 
    und das auch für alle Klassen $\Sigma^p_j, j> i$.
    Aus dieser Inklusion folgt somit direkt, dass die PH nur $i$ Stufen besitzt, und auf die Stufe $i$ kollabiert.
\end{proof}

Eine weitere Möglichkeit, sodass die PH kollabiert, besteht dann, wenn eine für eine Stufe $\Sigma^p_i = \Pi^p_i$ gilt:

\begin{theorem}
    Sei $\Sigma^p_i = \Pi^p_i$ für eine Stufe $i, i \geq 0$. Dann kollabiert die PH auf die Stufe $i$.
\end{theorem}

\begin{proof}[Beweis]
    Nach Voraussetzung gibt es eine Sprache $L$, sodass für jede Eingabe $x \in \{0,1\}^*$ eine Funktion mit 
    $$
    x \in L \Leftrightarrow \exists u_1 \in \{0,1\}^* \forall u_2 \in \{0,1\}^* ... Q_{i+1}u_{i + 1} \in \{0,1\}^* : M(x, u_1, ..., u_{i+1}) = 1
    $$
    existiert.
    Nun wird eine weitere Sprache $L'$ wie folgt konstruiert:
    $$
    \langle x, u_1 \rangle \in L' \Leftrightarrow \forall u_2 \in \{0,1\}^* \exists u_3 \in \{0,1\}^* ... Q_{i+1}u_{i + 1} \in \{0,1\}^* : M(x, u_1, ..., u_{i+1}) = 1
    $$
    Offensichtlich ist $L \in \Sigma^p_i$ und $L' \in \Pi^p_i$.
\end{proof}