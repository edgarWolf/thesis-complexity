\chapter{Alternierung} \label{chapter: Alternierung}

\section{Definition} \label{section: Definition}
Die Alternierung in der Komplexitätstheorie meint eine Verallgemeinerung des Nichtdeterminusmus. Mittels dieser Verallgemeinerung 
können diverse Beweise simplifiziert werden, und Verbindungen zu anderen Komplexitätsklassen gezogen werden \cite{sipser_introduction_2012} \cite{chandra_alternation_1976}. Ein Algorithmus, der auf Alternierung basiert,
kann wie ein gewöhnlicher nichtdeterministischer Algorithmus bei jedem Berechnungsschritt in mehrere Zustände überführt werden. Der Unterschied liegt hier bei dem Modus der Akzeptanz:
Während ein nichtdeterministischer akzeptiert, wenn mindestens ein Berechnugspfad akzeptiert, so tritt bei einem alternierenden Algorithmus eine Fallunterscheidung ein.
Die aktuelle Berechnug innerhalb des Algorithmus akzeptiert dann entweder, wenn wie beim Nichtdeterminusmus, mindestens ein Berechnugspfad akzeptiert, oder wenn alle Berechnugspfade akzeptieren.
Entscheidend für den Modus der Akzeptanz der Berechnung ist die Kennzeichnung des Berechnungsschrittes, der die Information enthält, unter welchem Modus dieser akzeptiert \cite{sipser_introduction_2012}.
Diese Kennzeichnung wir in der Literatur meist über die logischen Quantoren $\exists$ und $\forall$ vorgenommen, manche Autoren wie Sipser verwenden die Symbole $\land$ und $\lor$, für die Darstellung als logisches Und bzw. als logisches Oder.
Im ersten Fall akzeptiert die Berechnung, wenn mindestens ein Kind-Berechnugspfad akzeptiert, analog akzeptiert ein universeller Zustand genau dannn, wenn alle Kind-Berechnugspfade akzeptieren.
Eine Analogie hierfür könnte ein Prozess in einem Computer sein, der bei jedem Berechnungsschritt mehrere neue Prozesse erzeugt.
Jeder dieser Prozesse enthält eine Information darüber, ob er eine existenzielle oder universelle Berechnug durchführt. Der erzeugende Prozess gibt nun selber $1$ zurück, wenn er sich in einem existenziellen Zustand befindet
und mindestens ein Kind-Prozess mit $1$ antwortet, oder er sich in einem universellen Prozess befindet und alle Prozesse mit $1$ antworten \cite{sipser_introduction_2012}.
Wir können daher den Nichtdeterminusmus als einen Spezalfall der Alternierung auffassen, bei dem Berechungen ausschließlich existenziell durchgeführt werden \cite{chandra_alternation_1976}.

\section{Alternierende Turingmaschinen} \label{section: alternierende Turingmaschine}

\section{Komplexitätsklassen} \label{section: Komplexitätsklassen bei alternierenden TMs}