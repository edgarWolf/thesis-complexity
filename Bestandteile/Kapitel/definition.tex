\chapter{Polynomialzeithierarchie (PH)} \label{chapter: Polynomialzeithierarchie (PH)}
Die Komplexitätsklassen $P$, $NP$, und $coNP$ decken einige natürliche Entscheidungsprobleme nicht ab, die aber auch 
nicht $EXPTIME$ zuzuordnen sind. Für diese Probleme wurde die Polynomialzeithierarchie (abkürzend PH) eingeführt, sodass 
auch diese Probleme entsprechend auf ihre Komplexität untersucht werden können.
Die PH verfügt dabei über verschiedene Definitionen, die aber allesamt äquivalent sind.
% TODO: Hier wenn Zeit und Platz dasa mit EXACT-INDSET oder so einfügen


\section{Definition mit alternierenden Quantoren} \label{section: Definition PH mit alternierenden Quantoren}
Eine Möglichkeit der Definition besteht übr die der alternierenden Quantoren. Es wird dabei zwischen zwei Sprachdefinitionen unterschieden.
Sei dafür $p$ ein Polynom und $i \geq 1$. Dann ist eine Sprache $L$ in $\Sigma^p_i$, wenn es polynomiell zeitbeschränkte Turingmaschine $M$ gibt, 
sodass für jede Eingabe $x \in \{0,1\}^*$ gilt:
\begin{align*}
    x \in L \Leftrightarrow \exists u_1 \forall u_2 ... Q_i u_i : M(x, u_1, ..., u_i) = 1
\end{align*}
Analog ist eine Sprache $L$ in $\Pi^p_i$, wenn es eine polynomiell zeitbeschränkte Turingmaschine $M$ gibt, sodass für jede Eingabe $x \in \{0,1\}^*$ gilt:
\begin{align*}
    x \in L \Leftrightarrow \forall u_1 \exists u_2 ... Q_i u_i : M(x, u_1, ..., u_i) = 1
\end{align*}
Aus diesen Definitionen kann sofort gefolgert werden dass $\Sigma^p_1 = NP$ und $\Sigma^p_1 = coNP$, denn die boolsche Formel reduziert auf 
den ersten Quanotren entspricht genau der Definition der entsprechenden Komplexitätsklassen $NP$ bzw. $coNP$.
\section{Definition mit alternierenden Turingmaschinen} \label{section: Definition PH mit ATM}


\section{Definition mit Orakel-Turingmaschinen} \label{section: Definition PH mit Orakel-Turingmaschinen}