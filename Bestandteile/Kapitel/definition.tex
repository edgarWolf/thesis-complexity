\chapter{Polynomialzeithierarchie (PH)} \label{chapter: Polynomialzeithierarchie (PH)}
Die Polynomialzeithierarchie (kurz PH) bildet als Komplexitätsklasse Probleme ab, die sich nicht effizient mittels Nichtdeterminusmus lösen lassen, aber gleichzeitig 
nicht EXPTIME zuzuordnen sind. Die PH bietet einen Formalismus, um auch diese Probleme adäquat zu untersuchen und sie auf ihre Komplexität hin bewerten zu können.
Die PH verfügt dabei über verschiedene Definitionen, die aber allesamt äquivalent sind.
% TODO: Hier wenn Zeit und Platz das mit EXACT-INDSET oder so einfügen


\section{Definition mit alternierenden Quantoren} \label{section: Definition PH mit alternierenden Quantoren}
Eine Möglichkeit der Definition besteht übr die der alternierenden Quantoren. Es wird dabei zwischen zwei Sprachdefinitionen unterschieden.
Sei dafür $p$ ein Polynom und $i \geq 1$. Dann ist eine Sprache $L$ in $\Sigma^p_i$, wenn es polynomiell zeitbeschränkte Turingmaschine $M$ gibt, 
sodass für jede Eingabe $x \in \{0,1\}^*$ gilt \cite{rossman_complexity_2015}:
%todo: Die Erklärung der Quantifizierer und wann welcher Qunator hergenommen wir hinzufügen.%
\begin{align*}
    x \in L \Leftrightarrow \exists u_1 \in \{0,1\}^{p(|x|)} \forall u_2 \in \{0,1\}^{p(|x|)} ... Q_i u_i \in \{0,1\}^{p(|x|)} : M(x, u_1, ..., u_i) = 1
\end{align*}
wobei $Q_i = \exists$, wenn $i$ gerade ist, sonst $Q_i = \forall$. \\
Analog ist eine Sprache $L$ in $\Pi^p_i$, wenn es eine polynomiell zeitbeschränkte Turingmaschine $M$ gibt, sodass für jede Eingabe $x \in \{0,1\}^*$ gilt \cite{arora_computational_2009}:
\begin{align*}
    x \in L \Leftrightarrow \forall u_1 \in \{0,1\}^{p(|x|)} \exists u_2 \in \{0,1\}^{p(|x|)} ... Q_i u_i \in \{0,1\}^{p(|x|)} : M(x, u_1, ..., u_i) = 1
\end{align*}
wobei $Q_i = \forall$, wenn $i$ gerade ist, sonst $Q_i = \exists$.
Aus diesen Definitionen kann sofort gefolgert werden dass $\Sigma^p_1 = NP$ und $\Sigma^p_1 = coNP$, denn die boolsche Formel reduziert auf 
den ersten Quantoren entspricht genau der Definition der entsprechenden Komplexitätsklassen $NP$ bzw. $coNP$.
Die Polynomialzeithierarchie entspricht dann der Vereinigung über alle $\Sigma^p_i$, sowie der Vereinigung aller $\Pi^p_i$ \cite{sipser_introduction_2012}:
\begin{align*}
    \text{PH} = \bigcup_{i \geq 1} \Sigma^p_i = \bigcup_{i \geq 1} \Pi^p_i 
\end{align*}
\section{Definition mit alternierenden Turingmaschinen} \label{section: Definition PH mit ATM}
Eine äquivalente und mit der Definition über die alternierenden Quantoren verwandte Definition ist über alternierende Turingmaschinen möglich.
Das Konzept wurde bereits in \ref{section: alternierende Turingmaschine} erläutert und dient nun als Basis für die Definition der Polynomialzeithierarchie.
Sei dafür $M$ eine $f(n)$ zeitbeschränkte ATM und $i \geq 1$  mit $i - 1$ mit  Alternierungen. Dann entscheidet M die Sprache $\Sigma_i TIME(f(n))$, wenn der initiale Zustand ein existenzieller ist,
oder die Sprache $\Pi_i TIME(f(n))$, falls der erste Zustand ein universeller ist \cite{arora_computational_2009}.
Die Sprachen $\Sigma^p_i$ und $\Pi^p_i$ sind dann definiert als \cite{arora_computational_2009}:
\begin{align*}
    \Sigma^p_i = \bigcup_{c \in \mathbb{N}} \Sigma_i \text{TIME}(n^c) \\
    \Pi^p_i = \bigcup_{c \in \mathbb{N}} \Pi_i \text{TIME}(n^c)
\end{align*}
$\Sigma^p_i$ beschreibt also die Menge der Sprachen, die von einer polynomiell zeitbeschränkten ATM $M$ mit existenziellem Startzustand und höchstens $i-1$ Alternierungen entschieden werden.
$\Pi^p_i$ beschreibt analog die Menge der Sprachen, die von einer polynomiell zeitbeschränkten ATM $M$ mit universellem Startzustand und höchstens $i-1$ Alternierungen entschieden werden.
Die PH selbst ist wie in \ref{section: Definition PH mit alternierenden Quantoren} definiert als die Vereinigung aller $\Sigma^p_i$ und $\Pi^p_i$.

\section{Definition mit Orakel-Turingmaschinen} \label{section: Definition PH mit Orakel-Turingmaschinen}
Eine weitere Definition erfolgt über Turingmaschinen mit Zugriff auf ein Orakel. Dabei wird auf ein Orakel zugegriffen,
das ein vollständiges Problem einer Komplexitätsklasse entscheiden kann. Sei für die folgenden Ausführungen $\text{A}^\text{B}$, sodass eine A eine Komplexitätsklasse ist, deren Sprachen durch eine Turingmaschine mit Zugriff auf ein Orakel, welches alle Probleme aus B entscheiden kann, entscheidet. 
Dann ist die Polynomialzeithierarchie induktiv, für $i \geq 0$ definiert als \cite{rothe_komplexitatstheorie_2008}:
\begin{align*}
    & \Delta^p_0 = \Sigma^p_0 = \Pi^p_0 = \text{P} \\
    & \Delta^p_{i+1} = \text{P}^{\Sigma^p_i} \\
    &\Sigma^p_{i+1} = \text{NP}^{\Sigma^p_i} \\
    & \Pi^p_{i+1} = \text{coNP}^{\Sigma^p_i} \\
    & \text{PH} = \bigcup_{k \geq 0} \Sigma^p_k
\end{align*}
Der Unterschied zu den anderen zwei Definitionen besteht dabei über die Komplexitätsklassen $\Delta^p_i$, die die Komplexitätsklasse P
mit Zugriff auf Orakel betrachtet. Jedoch spielt diese Komplexitätsklasse in der strukturellen Betrachtung der Polynomialzeithierarchie keine große
Rolle und ist für die weiteren Ausführungen von nebensächlicher Relevanz. \\

\noindent Die obigen Definitionen sind äquivalent, was gezeigt werden kann, indem gezeigt wird dass $\Sigma^p_{i+1} = \text{NP}^{\Sigma^p_i \text{SAT}}$.
Dafür wird zuerst gezeigt, dass $\Sigma^p_2 \subseteq \text{NP}^{\text{SAT}}$. Sei dafür $L \in \Sigma^p_2$. Dann gilt: 
\begin{align*}
    x \in L \Leftrightarrow \exists u_1 \forall u_2 : M(x, u_1, u_2) = 1
\end{align*}
Dann kann $u_1$ nichtdeterministisch geraten werden, um dann das Orakel, welches SAT entscheiden kann, zu fragen, ob es ein $u_2$ gibt, sodass das Orakel die Eingabe nicht akzeptiert.
$M$ akzeptiert genau dann, wenn das Orakel nicht akzeptiert. \\
Nun wird gezeigt, dass $\text{NP}^{\text{SAT}} \subseteq \Sigma^p_2 $. Dann kann $L$ entschieden werden, indem es nichtdeterministisch Entscheidungen $y$ gibt, 
sowie Orakel-Anfragen $q_1, ..., q_k$ und Orakel-Antworten $a_1, ..., a_k$ sodass $M$ $x$ in polynomieller Zeit akzeptiert. Zusätzlich muss sichergestellt sein, 
dass die Antworten $a_i$ korrekte Antworten bezüglich der Anfragen $q_i$ sind, also dass $a_i = 1$ genau dann wenn  $q_i \in \text{SAT}$.